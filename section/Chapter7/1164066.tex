\section{Annisa Cahyani-1164066}
\subsection{Teori}
\begin{enumerate}
\item Mengapa File Teks Harus Dilakukan Tokenizer Besera Ilustrasi Gambar :
\begin{itemize}
\item Tokenizer :
\par Difungsikan untuk membuat vektor dari text. Lebih detailnya, tokenizer merupakan langkah pertama yang diperlukan dalam banyak tugas pemrosesan bahasa alami, seperti penghitungan kata.
\par
\par
\item Mengapa Text Harus Dilakukan Tokenizer ? :
\par Text harus dilakukan tokenizer agar dapat dirubah menjadi vektor dan dapat terbaca.
\par
\par
\item Ilustrasi Gambar : \ref{chapter-7-no-1-cahya}
\par
\begin{figure}[!hbtp]
\centering
\includegraphics[scale=0.2]{figures/Chapter 7/1164066/Teori/chapter-7-no-1-cahya.jpg}
\caption{Tokenizer - cahya}
\label{chapter-7-no-1-cahya}
\end{figure}
\par
\end{itemize}
\par
\par
\par
\par
\item Konsep Dasar K Fold Cross Validation Pada Dataset Komentar Youtube Pada Kode Listing Beserta Dengan Ilustrasi Gambar :
\begin{itemize}
\item Code		:
\lstinputlisting[firstline=8, lastline=20, caption=K Fold Cross Validation,label={lst:7.0}]{src/1164066/chapter-7-2-cahya.py}
\item Penjelasan	: 
\par Untuk kejelasan dari StartifiedKFold yang dicontohkan ialah digunakan dan berisikan presentasi sampel untuk setiap kelas yang ada ( youtube).
\par
\end{itemize}
\par
\par
\par
\item Jelaskan Apa Maksud Kode Program For Train Dan Test In Splits Dilengkapi Dengan Ilustrasi Gambar :
\begin{itemize}
\item Penjelasan	:
\par Kode Program For Train dan Test In Splits sendiri digunakan ataupun difungsikan untuk pengujian. Pegujiannya yaitu menguji apakah setiap data pada dataset yang dieksekusi sudah di split.
\par 
\par
\end{itemize}
\par
\item Apa Maksud Kode Program\emph{train\_content = d['CONTENT'].iloc[train\_idx]} dan \emph{test\_content = d['CONTENT'].iloc[test\_idx]}. Dilengkapi Dengan Ilustrasi Gambar :
\begin{itemize}
\item Penjelasan	:
\par Maksud dari code program tersebut ialah difungsikan dalam pengambilan data pada kolom atau index CONTENT. index CONTENT tersebut merupakan bagian dari train\_idx dan test\_idx.
\par
\par
\end{itemize}
\par
\item Apa Maksud Dari Fungsi Tokenizer = Tokenizer(num words=2000) Dan Tokenizer.fit on texts(train content), Dilengkapi Dengan Ilustrasi Gambar :
\begin{itemize}
\item Penjelasan	:
\par  Fungsi dari Tokenizer diatas ialah untuk melakukan vektorisasi kata tentunya. Fungsi tokenizer ini mengeksekusi jumlah data yang akan diubah sebesar 2000 kata. Kemudian untuk  \emph{tokenizer.fit\_on\_texts(train\_content)} digunakan untuk melakukan fit tokenizer.
\par
\par
\end{itemize}
\par
\item Apa Maksud Dari Fungsi code berikut ( \emph{d\_train\_inputs = tokenizer.texts\_to\_matrix(train\_content, mode='tfidf')} dan \emph{d\_test\_inputs = tokenizer.texts\_to\_matrix(test\_content, mode='tfidf')} ), Dilengkapi Dengan Ilustrasi Kode Dan Atau Gambar :
\begin{itemize}
\item Penjelasan	:
\par Dapat dikatakan bahwa maksud dari codingan diatas ialah untuk variabel d\_train\_inputs dimana akan melakukan tokenizer dari bentuk teks ke / menjadi matrix dari data train\_content menggunakan mode matrik yaitu tfidf.
\par
\par
\end{itemize}
\par
\item Jelaskan Apa Maksud Dari Fungsi Berikut ( \emph{d\_train\_inputs = d\_train\_inputs/np.amax(np.absolute(d\_train\_inputs))} dan \emph{d\_test\_inputs = d\_test\_inputs/np.amax(np.absolute(d\_test\_inputs))} ) Kemudian Dilengkapi Dengan Ilustrasi Gambar :
\begin{itemize}
\item Penjelasan : 
\par Berdasarkan code diatas, menjelaskan bahwa fungsi tersebut akan membagi matrix tfidf yang sudah dieksekusi sebelumnya dengan amax. Amaxnya berfungsi dalam pengembalian maksimum array atau maksimum sepanjang sumbu.
\par
\par
\end{itemize}
\par
\par
\par
\item Jelaskan Apa Maksud Dari \emph{d\_train\_outputs = np.utils.to\_categorical(d['CLASS'].iloc[train]} Dan \emph{d\_test\_outputs = np\_utils.to\_categorical(d['CLASS'].iloc[test\_idx]} Dalam Kode Program Dilengkapi Dengan Ilustrasi Gambar :
\begin{itemize}
\item Penjelasan : 
\par Yang dimaksudkan dari kode program tersebut dapat dijelaskan bahwa fungsnya ditujukan untuk melakukan one-hot encoding.
\par One-hot encoding diambil dari 'CLASS'  dengan neuron bernilai satu nol(1,0) atau nol satu(0,1).
\par
\par
\par
\end{itemize}
\par
\par
\item Jelaskan Maksud Dari Fungsi Di Listing 7.2. Gambarkan Ilustrasi Neural Networknya Dari Model Kode Tersebut.
\begin{itemize}
\item Code :
\lstinputlisting[firstline=8, lastline=20]{src/1164066/chapter-7-9-cahya.py}
\item Penjelasan : 
\par Berdasarkan code tersebut, dimaksudkan atau ditujukan untuk melakukan pemodelan dengan sequential.Terdapat 512 neuron inputan dengan input shape 2000 vektor. Selanjutnya model dilakukan aktivasi dengan fungsi 'relu' untuk pemotongan bobot  sebesar 50 persen.Setelah itu muncullah outputan yang diaktivasi menggunakan fungsi softmax.
\par
\par
\end{itemize}
\par
\par
\par
\par
\item Jelaskan Maksud Dari Fungsi Di Listing 7.3. Dengan Parameter Berikut :
\begin{itemize}
\item Code :
\lstinputlisting[firstline=8, lastline=20]{src/1164066/chapter-7-10-cahya.py}
\item Penjelasan : 
\par Berdasarkan code tersebut , dimaksudkan bahwa model yang telah dibuat akan dicompile dengan menggunakan algoritma optimisasi, fungsi loss, dan fungsi metrik.
\par
\par
\end{itemize}
\par
\par
\par
\item Apa itu Deep Learning :
\begin{itemize}
\item Penjelasan :
\par Deep learning merupakan sub bidang pembelajaran mesin yang berkaitan dengan algoritma.
\par
\end{itemize}
\item Apa itu Deep Neural Network Dan Apa Bedanya Dengan Deep Learning :
\begin{itemize}
\item Penjelasan Deep Neural Network : 
\par Deep neural network adalah jaringan syaraf dengan tingkat kompleksitas tertentu, jaringan syaraf dengan lebih dari dua lapisan.
\par
\item Perbedaan Deep Neural Network Dan Deep Learning :
\par Perbedaan antara deep neural network dan deep learning terletak pada kedalaman model. deep learning adalah frasa yang digunakan untuk jaringan saraf yang kompleks. Kompleksitas ini disebabkan oleh pola yang rumit tentang bagaimana informasi dapat mengalir di seluruh model.
\par
\par
\end{itemize}
\par
\par
\item Bagaimana Perhitungan Algoritma Dengan Ukuran Stride (NPM mod3+1)x(NPM mod3+1) Yang Terdapat Pada Max Pooling :
\begin{itemize}
\item Penjelasan :
\par Konvolusi pada sebuah gambar dilakukan dalam pemrosesan image untuk menerapkan operator yang mempunyai nilai output dari piksel gambar yang berasal dari kombinasi linear.
\par
\item Langkah-langkah Algoritma Konvulasi Sesuai NPM : \ref{chapter-7-13-cahya}
\par
\par
\begin{figure}[!hbtp]
\centering
\includegraphics[scale=0.52]{figures/Chapter 7/1164066/Teori/chapter-7-13-cahya.jpg}
\caption{Langkah Algoritma Konvolusi- cahya}
\label{chapter-7-13-cahya}
\end{figure}
\par
\par
\end{itemize}
\end{enumerate}


\subsection{Praktek}
Penjelasan Tugas Harian 12 ( No 1-20 ).
\begin{enumerate}
\item Jelaskan Kode Program Pada Blok \# In[1]. Jelaskan Arti Dari Setiap Baris Kode Yang Dibuat Dan Hasil Keluarannya Dari Komputer Sendiri.
\begin{itemize}
\item Code Yang Digunakan : \ref{lst:praktek-chapter-7-1-cahya}.
\lstinputlisting[firstline=1, lastline=19,caption=File praktek-chapter-7-1-cahya.py.py, label={lst:praktek-chapter-7-1-cahya}]{src/1164066/praktek-chapter-7-1-cahya.py}
\par
\par
\item Penjelasan Code Perbaris	: 
\begin{enumerate}
\item Baris Code 1	: Memasukkan / Mengimport file csv
\item Baris Code 2	: Memasukkan module image sebagai pil\_image dari library PIL
\item Baris Code 3	: Memasukkan / mengimport fungsi keras.processing.image 
\end{enumerate}
\par
\item Hasil : \ref{chapter-7-in-1-cahya}
\par
\par
\begin{figure}[!hbtp]
\centering
\includegraphics[scale=0.4]{figures/Chapter 7/1164066/Praktek/chapter-7-in-1-cahya.jpg}
\caption{Code Program Pada In [1] - cahya}
\label{chapter-7-in-1-cahya}
\end{figure}
\par
\par
\end{itemize}
\par
\par
\par
\item Jelaskan Kode Program Pada Blok \# In[2]. Jelaskan Arti Dari Setiap Baris Kode Yang Dibuat Dan Hasil Keluarannya Dari Komputer Sendiri.
\begin{itemize}
\item Code Yang Digunakan : \ref{lst:praktek-chapter-7-2-cahya}.
\lstinputlisting[firstline=1, lastline=19,caption=File praktek-chapter-7-2-cahya.py.py, label={lst:praktek-chapter-7-2-cahya}]{src/1164066/praktek-chapter-7-2-cahya.py}
\par
\par
\item Penjelasan Code Perbaris	: 
\begin{enumerate}
\item Baris Code 1	: Mendefinisikan variabel imgs tanpa parameter
\item Baris Code 2	: Mendefinisikan variabel classes tanpa parameter
\item Baris Code 3	: Membuka file HASYv2/hasy-data-labels.csv sebagai csvfile
\item Baris Code 4	: Mendefinisikan variabel csvreader yang memfungsikan pembacaan dari file csv
\item Baris Code 5	: Mendefinisikan variabel i dengan parameter 0
\item Baris Code 6	: Mengeksekusi baris dari pembacaan csv 
\item Baris Code 7	: Mengaplikasikan perintah "if" dengan ketentuan variabel i lebih besar dari angka 0
\item Baris Code 8	: Mendefinisikan variabel img yang mengubah image menjadi bentuk array (bilangan) dari file HASYv2
\item Baris Code 9	: Mendefinisikan variabel img tidak sama dengan nilai 255.0
\item Baris Code 10	: Mendefinisikan fungsi imgs.append dimana merupakan proses melampirkan atau menggabungkan data dengan file lain atau set data
\item Baris Code 11	: Mendefinisikan fungsi append kembali dari variabel classes dengan parameternya row[2].
\item Baris Code 12	: Mendefinisikan fungsi dimana i variabel i akan ditambah nilainya sehingga akan bernilai 1
\end{enumerate}
\par
\item Hasil : \ref{chapter-7-in-2-cahya}
\par
\par
\begin{figure}[!hbtp]
\centering
\includegraphics[scale=0.4]{figures/Chapter 7/1164066/Praktek/chapter-7-in-2-cahya.jpg}
\caption{Code Program Pada In [2] - cahya}
\label{chapter-7-in-2-cahya}
\end{figure}
\par
\par
\end{itemize}
\par
\par
\par
\item Jelaskan Kode Program Pada Blok \# In[3]. Jelaskan Arti Dari Setiap Baris Kode Yang Dibuat Dan Hasil Keluarannya Dari Komputer Sendiri.
\begin{itemize}
\item Code Yang Digunakan : \ref{lst:praktek-chapter-7-3-cahya}.
\lstinputlisting[firstline=1, lastline=19,caption=File praktek-chapter-7-3-cahya.py.py, label={lst:praktek-chapter-7-3-cahya}]{src/1164066/praktek-chapter-7-3-cahya.py}
\par
\par
\item Penjelasan Code Perbaris	: 
\begin{enumerate}
\item Baris Code 1	: Memasukkan  module random
\item Baris Code 2	: Melakukan pengocokan pada module random dengan parameter variabelnya imgs
\item Baris Code 3	: Memecah index dalam bentuk integer dengan mengkalikannilai 0,8 dan fungsi len yang akan mengembalikan jumlah item
\item Baris Code 4	: Mendefinisikan variabel train yang mengeksekusi imgs dengan pemecahan index awal pada data
\item Baris Code 5	: Mendefinisikan variabel test yang mengeksekusi imgs dengan pemecahan index akhir pada data
\end{enumerate}
\par
\item Hasil : \ref{chapter-7-in-3-cahya}
\par
\par
\begin{figure}[!hbtp]
\centering
\includegraphics[scale=0.4]{figures/Chapter 7/1164066/Praktek/chapter-7-in-3-cahya.jpg}
\caption{Code Program Pada In [3] - cahya}
\label{chapter-7-in-3-cahya}
\end{figure}
\par
\par
\end{itemize}
\par
\par
\par
\item Jelaskan Kode Program Pada Blok \# In[4]. Jelaskan Arti Dari Setiap Baris Kode Yang Dibuat Dan Hasil Keluarannya Dari Komputer Sendiri.
\begin{itemize}
\item Code Yang Digunakan : \ref{lst:praktek-chapter-7-4-cahya}.
\lstinputlisting[firstline=1, lastline=19,caption=File praktek-chapter-7-4-cahya.py.py, label={lst:praktek-chapter-7-4-cahya}]{src/1164066/praktek-chapter-7-4-cahya.py}
\par
\par
\item Penjelasan Code Perbaris	: 
\begin{enumerate}
\item Baris Code 1	: Memasukkan / Mengimport library numpy sebagai np
\item Baris Code 2	: Mendefinisikan variabel train\_input dimana mengubah input menjadi sebuah array dari np dengan menggunakan fungsi list untuk mengkoleksikan data yang dipilih dan dapat diubah. 
\item Baris Code 3	: Mendefinisikan variabel test\_input dengan fungsi yang sama seperti train\_input yang membedakan hanya datanya / inputan yang diproses berasal dari variabel test
\item Baris Code 4	: Mendefinisikan variabel train\_output dimana mengubah keluaran menjadi sebuah array dari np dengan menggunakan fungsi list untuk mengkoleksikan data yang dipilih dan dapat diubah. 
\item Baris Code 5	: Mendefinisikan variabel test\_output dengan fungsi yang sama seperti train\_output yang membedakan hanya datanya / inputan yang diproses berasal dari variabel test
\end{enumerate}
\par
\item Hasil : \ref{chapter-7-in-4-cahya}
\par
\par
\begin{figure}[!hbtp]
\centering
\includegraphics[scale=0.4]{figures/Chapter 7/1164066/Praktek/chapter-7-in-4-cahya.jpg}
\caption{Code Program Pada In [4] - cahya}
\label{chapter-7-in-4-cahya}
\end{figure}
\par
\par
\end{itemize}
\par
\par
\par
\item Jelaskan Kode Program Pada Blok \# In[5]. Jelaskan Arti Dari Setiap Baris Kode Yang Dibuat Dan Hasil Keluarannya Dari Komputer Sendiri.
\begin{itemize}
\item Code Yang Digunakan : \ref{lst:praktek-chapter-7-5-cahya}.
\lstinputlisting[firstline=1, lastline=19,caption=File praktek-chapter-7-5-cahya.py.py, label={lst:praktek-chapter-7-5-cahya}]{src/1164066/praktek-chapter-7-5-cahya.py}
\par
\par
\item Penjelasan Code Perbaris	: 
\begin{enumerate}
\item Baris Code 1	: Memasukkan modul / fungsi LabelEncoder dari sklearn.processing yang digunakan untuk menormalkan label dimana label encoder hanya didefinisikan dengan nilai antara 0 dan -1.
\item Baris Code 2	: Memasukkan modul / fungsi OneHotEncoder dari sklearn.processing yang digunakan untuk mendefinisikan fitur input dimana mengambil nilai dalam kisaran 0
\end{enumerate}
\par
\item Hasil : \ref{chapter-7-in-5-cahya}
\par
\par
\begin{figure}[!hbtp]
\centering
\includegraphics[scale=0.4]{figures/Chapter 7/1164066/Praktek/chapter-7-in-5-cahya.jpg}
\caption{Code Program Pada In [5] - cahya}
\label{chapter-7-in-5-cahya}
\end{figure}
\par
\par
\end{itemize}
\par
\par
\par
\item Jelaskan Kode Program Pada Blok \# In[6]. Jelaskan Arti Dari Setiap Baris Kode Yang Dibuat Dan Hasil Keluarannya Dari Komputer Sendiri.
\begin{itemize}
\item Code Yang Digunakan : \ref{lst:praktek-chapter-7-6-cahya}.
\lstinputlisting[firstline=1, lastline=19,caption=File praktek-chapter-7-6-cahya.py.py, label={lst:praktek-chapter-7-6-cahya}]{src/1164066/praktek-chapter-7-6-cahya.py}
\par
\par
\item Penjelasan Code Perbaris	: 
\begin{enumerate}
\item Baris Code 1	: Mendefinisikan variabel label\_encoder dengan penerapan modul / fungsi dari LabelEncoder
\item Baris Code 2	: Mendefinisikan variabel integer\_encoded dengan penerapan fungsi label\_encoder.fit\_transform (ekstrasi fitur object ) dari variabel classes dimana akan mengembalikan beberapa data yang diubah kembali
\end{enumerate}
\par
\item Hasil : \ref{chapter-7-in-6-cahya}
\par
\par
\begin{figure}[!hbtp]
\centering
\includegraphics[scale=0.4]{figures/Chapter 7/1164066/Praktek/chapter-7-in-6-cahya.jpg}
\caption{Code Program Pada In [6] - cahya}
\label{chapter-7-in-6-cahya}
\end{figure}
\end{itemize}r
\item Jelaskan Kode Program Pada Blok \# In[7]. Jelaskan Arti Dari Setiap Baris Kode Yang Dibuat Dan Hasil Keluarannya Dari Komputer Sendiri.
\begin{itemize}
\item Code Yang Digunakan : \ref{lst:chapter-7-7-cahya}.
\lstinputlisting[firstline=1, lastline=19,caption=File chapter-7-7-cahya.py, label={lst:chapter-7-7-cahya]{src/1164066/chapter-7-7-cahya.py}
\par
\par
\item Penjelasan Code Perbaris	: 
\begin{enumerate}
\item Baris Code 1	: Membuat variabel onehot\_encoder yang memanggil fungsi OneHotEncoder tanpa mengembalikan matriks karena sparse=false.
\item Baris Code 2	: Membuat variabel integer\_encoded memanggil variabel integer\_encoded pada kode program 6 untuk dieksekusi memberikan bentuk baru ke array tanpa mengubah datanya 
\item Baris Code 3	: Onehotencoding melakukan fitting pada integer\_encoded.
\end{enumerate}
\item Hasil : \ref{chapter-7-in-7-cahya}
\par
\par
\begin{figure}[!hbtp]
\centering
\includegraphics[scale=0.4]{figures/chapter-7-in-7-cahya.jpg}
\caption{Code Program Pada In [7] - cahya}
\label{chapter-7-in-7-cahya}
\end{figure}
\par
\par
\end{itemize}
\par
\par
\par
\item Jelaskan Kode Program Pada Blok \# In[8]. Jelaskan Arti Dari Setiap Baris Kode Yang Dibuat Dan Hasil Keluarannya Dari Komputer Sendiri.
\begin{itemize}
\item Code Yang Digunakan : \ref{lst:chapter-7-8-cahya}.
\lstinputlisting[firstline=1, lastline=19,caption=File chapter-7-8-cahya.py, label={lst:chapter-7-8-fcahya]{src/1164066/chapter-7-8-cahya.py}
\par
\par
\item Penjelasan Code Perbaris	: 
\begin{enumerate}
\item Baris Code 1	: Mendefinisikan dan Membuat variabel train\_output\_int yang mengeksekusi label\_encoder dengan mengubah nilai dari parameter variabel train\_output.
\item Baris Code 2	: Mendefinisikan dan Membuat variabel train\_output yang mengeksekusi variabel onehot\_encoder dari kode program 7 dengan mengubah nilai dari variabel parameter train\_output\_int yang datanya sudah diubah kedalam bentuk array 
\item Baris Code 3	: Mendefinisikan dan Membuat variabel test\_output\_int yang mengeksekusi label\_encoder dengan mengubah nilai dari parameter variabel test\_output.
\item Baris Code 4	: Mendefinisikan dan Membuat variabel test\_output yang mengeksekusi variabel onehot\_encoder dari kode program 7 dengan mengubah nilai dari variabel parameter test\_output\_int yang datanya sudah diubah kedalam bentuk array 
\item Baris Code 5	: Mendefinisikan dan Membuat variabel num\_classes untuk mengetahui jumlah class dari lebel\_encoder
\item Baris Code 6	: Menampilkan hasil dari variabel num\_classes
\end{enumerate}
\item Hasil : \ref{chapter-7-in-8-cahya}
\par
\par
\begin{figure}[!hbtp]
\centering
\includegraphics[scale=0.4]{figures/chapter-7-in-8-cahya.jpg}
\caption{Code Program Pada In [8] - cahya}
\label{chapter-7-in-8-cahya}
\end{figure}
\par
\par
\end{itemize}
\par
\par
\par
\item Jelaskan Kode Program Pada Blok \# In[9]. Jelaskan Arti Dari Setiap Baris Kode Yang Dibuat Dan Hasil Keluarannya Dari Komputer Sendiri.
\begin{itemize}
\item Code Yang Digunakan : \ref{lst:chapter-7-9-cahya}.
\lstinputlisting[firstline=1, lastline=19,caption=File chapter-7-9-cahya.py, label={lst:chapter-7-9-cahya]{src/1164066/chapter-7-9-cahya.py}
\par
\par
\item Penjelasan Code Perbaris	: 
\begin{enumerate}
\item Baris Code 1	: Melakukan importing fungsi model sequential dari library keras.
\item Baris Code 2	: Melakukan importing fungsi layer dense, dropout, dan flatten dari library keras.
\item Baris Code 3	: Melakukan importing fungsi layer Conv2D dan MaxPooling2D dari library keras.
\end{enumerate}
\par
\item Hasil : \ref{chapter-7-in-9-cahya}
\par
\par
\begin{figure}[!hbtp]
\centering
\includegraphics[scale=0.4]{figures/chapter-7-in-9-cahya.jpg}
\caption{Code Program Pada In [9] - cahya}
\label{chapter-7-in-9-cahya}
\end{figure}
\par
\par
\end{itemize}
\par
\par
\par
\item Jelaskan Kode Program Pada Blok \# In[10]. Jelaskan Arti Dari Setiap Baris Kode Yang Dibuat Dan Hasil Keluarannya Dari Komputer Sendiri.
\begin{itemize}
\item Code Yang Digunakan : \ref{lst:chapter-7-10-cahya}.
\lstinputlisting[firstline=1, lastline=19,caption=File chapter-7-10-cahya.py, label={lst:chapter-7-10-cahya}]{src/1164066/chapter-7-10-cahya.py}
\par
\par
\item Penjelasan Code Perbaris : 
\begin{enumerate}
\item Baris Code 1 : Merealisasikan dan Mengfungsikan pemodelan Sequential
\item Baris Code 2 : Mendefinisikan fungsi konvolusi 2D dengan 32 filter konvolusi 3x3 dengan algoritam activation relu
\item Baris Code 3 : Dilanjutkan dengan data dari train\_input
\item Baris Code 4	: Penambahan fungsi max pooling dengan matriks 2x2
\item Baris Code 5	: Dilakukan konvolusi 2D dengan 32 filter konvolusi 3x3 dengan algoritam activation relu.
\item Baris Code 6	: Memfungsikan kembali fungsi max pooling dengan matriks 2x2
\item Baris Code 7	: Memfungsikan flatten untuk mengembalikan salinan array
\item Baris Code 8	: Mendefinisikan inputan dengan 1024 neuron dan menggunakan algoritma tanh
\item Baris Code 9	: Mendefinisikan fungsi Dropout dimana terdiri dari pengaturan secara acak tingkat pecahan unit, yang membantu mencegah overfitting sebesar 50\%.
\item Baris Code 10	: Untuk output layer menggunakan data dari variabel num\_classes dengan fugsi activationnya softmax.
\item Baris Code 11	: Mendefinisikan konfigurasi dari proses pembelajaran, yang dilakukan melalui metode compile.
\item Baris Code 12	: Mencetak  representasi ringkasan modelnya
\end{enumerate}
\par
\item Hasil : \ref{chapter-7-in-10-cahya}
\par
\par
\begin{figure}[!hbtp]
\centering
\includegraphics[scale=0.4]{figures/chapter-7-in-10-cahya.jpg}
\caption{Code Program Pada In [10] - cahya}
\label{chapter-7-in-10-cahya}
\end{figure}
\par
\par
\end{itemize}
\par
\par
\par
\item Jelaskan Kode Program Pada Blok \# In[11]. Jelaskan Arti Dari Setiap Baris Kode Yang Dibuat Dan Hasil Keluarannya Dari Komputer Sendiri.
\begin{itemize}
\item Code Yang Digunakan : \ref{lst:chapter-7-11-cahya}.
\lstinputlisting[firstline=1, lastline=19,caption=File chapter-7-11-cahya.py, label={lst:chapter-7-11-cahya}]{src/1164066/chapter-7-11-cahya.py}
\par
\par
\item Penjelasan Code Perbaris : 
\begin{enumerate}
\item Baris Code 1 : Memasukkan / Mengimport library keras.callbacks dimana digunakan dalam penulisan log untuk TensorBoard
\item Baris Code 2 : Membuat variabel tenserboard yang mendefinisikan fungsi TensorBoard pada keras.callbacks yang digunakan sebagai alat visualisasi yang disediakan dengan TensorFlow. Kemudian untuk fungsi log\_dir memanggil data yaitu './logs/mnist-style'
\end{enumerate}
\par
\item Hasil : \ref{chapter-7-in-11-cahya}
\par
\par
\begin{figure}[!hbtp]
\centering
\includegraphics[scale=0.4]{figures/chapter-7-in-11-cahya.jpg}
\caption{Code Program Pada In [11] - cahya}
\label{chapter-7-in-11-cahya}
\end{figure}
\par
\par
\end{itemize}
\par
\par
\par
\item Jelaskan Kode Program Pada Blok \# In[12]. Jelaskan Arti Dari Setiap Baris Kode Yang Dibuat Dan Hasil Keluarannya Dari Komputer Sendiri.
\begin{itemize}
\item Code Yang Digunakan : \ref{lst:chapter-7-12-cahya}.
\lstinputlisting[firstline=1, lastline=19,caption=File chapter-7-12-cahya.py, label={lst:chapter-7-12-cahya}]{src/1164066/chapter-7-12-cahya.py}
\par
\par
\item Penjelasan Code Perbaris : 
\begin{enumerate}
\item Baris Code 1 : Menerapkan fungsi model.fit yang didalamnya memproses train\_input, train\_output
\item Baris Code 2 : Selanjutnya pada penerapan fungsi yang sama difungsikan batch\_size apabila batch\_sizenya tidak ditemukan maka otomatis akan dijadikan nilai 32
\item Baris Code 3 : Pada penerapan fungsi yang sama, difungsikan epochs dimana perulangan dari berapa kali nilai yang digunakan untuk data, dan jumlahnya ialah 10
\item Baris Code 4 : Membuat fungsi verbose dimana digunakan sebagai opsi untuk menghasilkan informasi logging dari data yang ditentukan dengan nilai 2
\item Baris Code 5 : Membuat fungsi validation\_split untuk memecah nilai dari perhitungan validasinya sebesar 0,2
\item Baris Code 6 : Membuat fungsi callsbacks dengan parameternya yang mengeksekusi tensorboard 
\item Baris Code 7 : Membuat variabel score dengan fungsi evaluate dari model yang ada dengan parameter test\_input, tst\_output dan verbose=2 dimana memprediksi output untuk input yang diberikan
\item Baris Code 8 : Mencetak score optimasi dari test dengan ketentuan nilai parameter 0
\item Baris Code 9 : Mencetak score akurasi dari test dengan ketentuan nilai parameter 1
\end{enumerate}
\par
\item Hasil : \ref{chapter-7-in-12-cahya}
\par
\par
\begin{figure}[!hbtp]
\centering
\includegraphics[scale=0.4]{figures/chapter-7-in-12-cahya.jpg}
\caption{Code Program Pada In [12] - cahya}
\label{chapter-7-in-12-cahya}
\end{figure}
\par
\par
\end{itemize}
\par
\par
\par
\par
\end{enumerate}


\subsection{Penanganan Error}
Penjelasan Tugas Harian 12 ( No 1-20 ).
\par
\par
\par
\par
\par
\item Jelaskan Kode Program Pada Blok \# In[13]. Jelaskan Arti Dari Setiap Baris Kode Yang Dibuat Dan Hasil Keluarannya Dari Komputer Sendiri.
\begin{itemize}
\item Code Yang Digunakan : \ref{lst:chapter-7-13-cahya}.
\lstinputlisting[firstline=1, lastline=19,caption=File chapter-7-13-cahya.py, label={lst:chapter-7-13-cahya}]{src/1164066/chapter-7-13-cahya.py}
\par
\par
\item Penjelasan Code Perbaris : 
\begin{enumerate}
\item Baris Code 1	: Melakukan impor modul time dari python anaconda
\item Baris Code 2	: Membuat variabel result berisikan array kosong.
\item Baris Code 3	: Menggunakan convolution 2D
\item Baris Code 4	: Mendefinisikan dense\_size
\item Baris Code 5	: Mendefinsikan drop\_out
\item Baris Code 6	: Menerapkan pemodelan Sequential
\item Baris Code 7	: Perlu memasukkan bentuk input apabila data tersebut merupakan layer pertama
\item Baris Code 8	: Apabila tidak maka hanya menambahkan layer
\item Baris Code 9	: Ketika penambahan layer konvolusi selesai, maka dilakukan hal yang sama dengan max pooling.
\item Baris Code 10	: Kemudian meratakan atau flatten pada data dan menambahkan dense size ukuran apa pun yang berasal dari dense\_size
\item Baris Code 11	: Apabila dropout digunakan, dilakukan penambahan layer dropout. Dimisalkan 50\%, bahwa setiap kali ia memperbarui bobot setelah setiap batch, ada peluang 50\% untuk setiap bobot yang tidak akan diperbarui pada data tersebut
\item Baris Code 12	: Melakukan penempatan hasil drouput di antara dua lapisan untuk melindunginya dari overfitting.
\item Baris Code 13	: Lapisan terakhir akan selalu menjadi jumlah kelas karena itu harus
\item Baris Code 14	: Mengatur direktori log yang berbeda untuk TensorBoard sehingga dapat membedakan konfigurasi yang berbeda.
\item Baris Code 15	: Membuat variabel start yang akan memanggil modul time
\item Baris Code 16	: Melakukan dan memfungsikan fit
\item Baris Code 17	: Melakukan dan memfungsikan scoring dengan .evaluate yang akan menampilkan data loss dan accuracy dari model
\item Baris Code 18	: Untuk end merupakan variabel untuk melihat waktu akhir pada saat pemodelan berhasil dilakukan.
\item Baris Code 19	: Mencetak hasil dari run skrip
\end{enumerate}
\par
\item Hasil : \ref{chapter-7-13-cahya}
\par
\par
\begin{figure}[!hbtp]
\centering
\includegraphics[scale=0.4]{figures/chapter-7-13-cahya.jpg}
\caption{Code Program Pada  [13] - cahya}
\label{chapter-7-13-cahya}
\end{figure}
\par
\par
\end{itemize}
\par
\par
\par
\par
\end{enumerate}
\par
\par
\par
\item Jelaskan Kode Program Pada Blok \# In[14]. Jelaskan Arti Dari Setiap Baris Kode Yang Dibuat Dan Hasil Keluarannya Dari Komputer Sendiri.
\begin{itemize}
\item Code Yang Digunakan : \ref{lst:chapter-7-14-cahya}.
\lstinputlisting[firstline=1, lastline=19,caption=File chapter-7-14-cahya.py, label={lst:chapter-7-14-cahya}]{src/1164066/chapter-7-14-cahya.py}
\par
\par
\item Penjelasan Code Perbaris : 
\begin{enumerate}
\item Baris Code 1	: Memfungsikan pemodelan Sequential pada code
\item Baris Code 2	: Memfungsikan Convolution 2D dengan dmensi 32 untuk layer pertama,dan ukuran matriks 3x3 dengan function aktivasi yaitu relu
\item Baris Code 3	: Memfungsikan max pooling 2D dengan ukuran matriks 2x2
\item Baris Code 4	: Melakukan Convolusi lagi dengan kriteria yang sama tanpa menambahkan input untuk di layer kedua
\item Baris Code 5	: Menambahkan fungsi Max pooling 2D dengan ukuran poolnya yaitu 2x2.
\item Baris Code 6	: Mendefinisikan fungsi flatten dimana diugnakan untuk meratakan inputan yang ada dan mengembalikan salinan array
\item Baris Code 7	: Menambahkan dense input sebanyak 128 neuron dengan menggunakan function aktivasi tanh.
\item Baris Code 8	: Menambahkan fungsi dropout sebanyak 50\% untuk menghindari overfitting data
\item Baris Code 9	: Memfungsikan dense pada model untuk output
\item Baris Code 10	: Mengcompile model yang telah dibuat dan didefinisikan
\item Baris Code 11	: Mencetak ringkasan dari pemodelan yang dilakukan
\end{enumerate}
\par
\item Hasil : \ref{chapter-7-14-cahya}
\par
\par
\begin{figure}[!hbtp]
\centering
\includegraphics[scale=0.4]{figures/chapter-7-14-cahya.jpg}
\caption{Code Program Pada  [14] - cahya}
\label{chapter-7-14-cahya}
\end{figure}
\par
\par
\end{itemize}
\par
\par
\par
\par
\end{enumerate}
\par
\par
\par
\item Jelaskan Kode Program Pada Blok \# In[15]. Jelaskan Arti Dari Setiap Baris Kode Yang Dibuat Dan Hasil Keluarannya Dari Komputer Sendiri.
\begin{itemize}
\item Code Yang Digunakan : \ref{lst:chapter-7-15-cahya}.
\lstinputlisting[firstline=1, lastline=19,caption=File chapter-7-15-cahya.py, label={lst:chapter-7-15-cahya}]{src/1164066/chapter-7-15-cahya.py}
\par
\par
\item Penjelasan Code Perbaris : 
\begin{enumerate}
\item Baris Code 1	: Memfungsikan dan Melakukan fit dengan join data train\_input dan test\_input agar dapat dilakukan pelatihan.
\item Baris Code 2	: Memfungsikan kembali np.concenate dengan data dari train\_output dan test\_output
\item Baris Code 3	: Dilanjutkan dengan penerapan batch\_size sebesar 32, perulangan data 10 kali (epochs) dan verbose = 2
\end{enumerate}
\par
\item Hasil : \ref{chapter-7-15-cahya}
\par
\par
\begin{figure}[!hbtp]
\centering
\includegraphics[scale=0.4]{figures/chapter-7-15-cahya.jpg}
\caption{Code Program Pada  [15] - cahya}
\label{chapter-7-15-cahya}
\end{figure}
\par
\par
\end{itemize}
\par
\par
\par
\par
\end{enumerate}
\par
\par
\par
\item Jelaskan Kode Program Pada Blok \# In[16]. Jelaskan Arti Dari Setiap Baris Kode Yang Dibuat Dan Hasil Keluarannya Dari Komputer Sendiri.
\begin{itemize}
\item Code Yang Digunakan : \ref{lst:chapter-7-16-cahya}.
\lstinputlisting[firstline=1, lastline=19,caption=File chapter-7-16-cahya.py, label={lst:chapter-7-16-cahya}]{src/1164066/chapter-7-16-cahya.py}
\par
\par
\item Penjelasan Code Perbaris : 
\begin{enumerate}
\item Baris Code 1: Save model yang telah di latih dengan nama mathsymbols.model
\end{enumerate}
\par
\item Hasil : \ref{chapter-7-16-cahya}
\par
\par
\begin{figure}[!hbtp]
\centering
\includegraphics[scale=0.4]{figures/chapter-7-16-cahya.jpg}
\caption{Code Program Pada  [16] - cahya}
\label{chapter-7-16-cahya}
\end{figure}
\par
\par
\end{itemize}
\par
\par
\par
\par
\end{enumerate}
\par
\par
\par
\item Jelaskan Kode Program Pada Blok \# In[17]. Jelaskan Arti Dari Setiap Baris Kode Yang Dibuat Dan Hasil Keluarannya Dari Komputer Sendiri.
\begin{itemize}
\item Code Yang Digunakan : \ref{lst:chapter-7-17-cahya}.
\lstinputlisting[firstline=1, lastline=19,caption=File chapter-7-17-cahya.py, label={lst:chapter-7-17-cahya}]{src/1164066/chapter-7-17-cahya.py}
\par
\par
\item Penjelasan Code Perbaris : 
\begin{enumerate}
\item Baris Code 1: Menyimpan label enkoder (untuk membalikkan one-hot encoder) dengan nama classes.npy
\end{enumerate}
\par
\item Hasil : \ref{chapter-7-17-cahya}
\par
\par
\begin{figure}[!hbtp]
\centering
\includegraphics[scale=0.4]{figures/chapter-7-17-cahya.jpg}
\caption{Code Program Pada  [17] - cahya}
\label{chapter-7-17-cahya}
\end{figure}
\par
\par
\end{itemize}
\par
\par
\par
\par
\end{enumerate}
\par
\par
\par
\item Jelaskan Kode Program Pada Blok \# In[18]. Jelaskan Arti Dari Setiap Baris Kode Yang Dibuat Dan Hasil Keluarannya Dari Komputer Sendiri.
\begin{itemize}
\item Code Yang Digunakan : \ref{lst:chapter-7-18-cahya}.
\lstinputlisting[firstline=1, lastline=19,caption=File chapter-7-18-cahya.py, label={lst:chapter-7-18-cahya}]{src/1164066/chapter-7-18-cahya.py}
\par
\par
\item Penjelasan Code Perbaris : 
\begin{enumerate}
\item Baris Code 1: Memasukkan  models dari librari Keras ( keras.models )
\item Baris Code 2: Membuat variabel model2 yang akan difungsikan untuk memanggil model
\item Baris Code 3: Mencetak ringkasan dari hasil pemodelan pada variabel model2
\end{enumerate}
\par
\item Hasil : \ref{chapter-7-18-cahya}
\par
\par
\begin{figure}[!hbtp]
\centering
\includegraphics[scale=0.4]{figures/chapter-7-18-cahya.jpg}
\caption{Code Program Pada  [18] - cahya}
\label{chapter-7-18-cahya}
\end{figure}
\par
\par
\end{itemize}
\par
\par
\par
\par
\end{enumerate}
\par
\par
\par
\item Jelaskan Kode Program Pada Blok \# In[19]. Jelaskan Arti Dari Setiap Baris Kode Yang Dibuat Dan Hasil Keluarannya Dari Komputer Sendiri.
\begin{itemize}
\item Code Yang Digunakan : \ref{lst:chapter-7-19-cahya}.
\lstinputlisting[firstline=1, lastline=19,caption=File chapter-7-19-cahya.py, label={lst:chapter-7-19-cahya}]{src/1164066/chapter-7-19-cahya.py}
\par
\par
\item Penjelasan Code Perbaris : 
\begin{enumerate}
\item Baris Code 1: Melakukan pemanggilan fungsi LabelEncoder
\item Baris Code 2: Membuat variabel label\_encoder akan memanggil class
\item Baris Code 3: Menerapkan function predict
\item Baris Code 4 : Fungsi prediksi dibuat variabel newing yang akan memproses perubahan format gambar menjadi array
\item Baris Code 5: Variabel newing didefinisikan tidak sama dengan nilai 255.0
\item Baris Code 5: Kemudian untuk variabel prediction akan melakukan prediksi untuk model2 dengan reshape variabel newimg
\item Baris Code 5: Membuat variabel inverted yang akan mencari nilai tertinggi output dari hasil prediksi
\item Baris Code 6: Menampilkan hasil dari variabel prediction dan inverted
\end{enumerate}
\par
\item Hasil : \ref{chapter-7-19-cahya}
\par
\par
\begin{figure}[!hbtp]
\centering
\includegraphics[scale=0.4]{figures/chapter-7-19-cahya.jpg}
\caption{Code Program Pada  [19] - cahya}
\label{chapter-7-19-cahya}
\end{figure}
\par
\par
\end{itemize}
\par
\par
\par
\par
\end{enumerate}
\par
\par
\par
\item Jelaskan Kode Program Pada Blok \# In[20]. Jelaskan Arti Dari Setiap Baris Kode Yang Dibuat Dan Hasil Keluarannya Dari Komputer Sendiri.
\begin{itemize}
\item Code Yang Digunakan : \ref{lst:chapter-7-20-cahya}.
\lstinputlisting[firstline=1, lastline=19,caption=File chapter-7-19-cahya.py, label={lst:chapter-7-20-cahya}]{src/1164066/chapter-7-20-cahya.py}
\par
\par
\item Penjelasan Code Perbaris : 
\begin{enumerate}
\item Baris Code 1: Menerapkan  prediksi dari pelatihan dari gambar v2-00010.png
\item Baris Code 2: Menerapkan  prediksi dari pelatihan dari gambar v2-00500.png
\item Baris Code 3: Menerapkan  prediksi dari pelatihan dari gambar v2-00700.png
\end{enumerate}
\par
\item Hasil : \ref{chapter-7-20-cahya}
\par
\par
\begin{figure}[!hbtp]
\centering
\includegraphics[scale=0.4]{figures/chapter-7-20-cahya.jpg}
\caption{Code Program Pada  [20] - cahya}
\label{chapter-7-20-cahya}
\end{figure}
\par
\par
\end{itemize}
\par
\par
\par
\par
\end{enumerate}

\subsection{Penanganan Error}

\subsection{Penanganan Error - Annisa Cahyani}
\begin{enumerate}
\item Error 1	:
\begin{itemize}
\item Screenshoot Error : \ref{chapter-7-error-cahya}
\par
\par
\begin{figure}[!hbtp]
\centering
\includegraphics[scale=0.2]{figures/chapter-7-error-cahya.jpeg}
\caption{erorr-cahya}
\label{chapter-7-error-cahya}
\end{figure}
\par
\item Code Error :
\begin{lstlisting}
NameError: name 'model' is not defined
\end{lstlisting}
\item Penanganan Error :
\begin{enumerate}
\item Pertama-tama pastikan salahnya seperti apa ( model dan jenis errornya )
\item Kemudian, silahkan proses error tersebut dengan cara yang sesuai
\item Berdasarkan error maka penyelesaiannya ialah melakukan pendefinisian variabel model sehingga code dapat dijalankan 
\end{enumerate}
\end{itemize}
\end{enumerate}


